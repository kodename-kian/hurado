\subsection*{Problem Statement}

This is a sample batch task with a custom checker / grader. The task is to output the input strings in uppercase.

\subsection*{Input Format}

The first line of input contains an integer \(n\), the number of lines to upper case.

Each of the next \(n\) lines contains a string \(s\).

\subsection*{Output Format}

Output \(n\) lines, each containing the string \(s\) in uppercase.

\subsection*{Scoring}

Your score is proportional to how many strings you output correctly.

There will be exactly two subtasks. Each subtask will have two test files.

The first subtask will have two files with \(n = 4\) and \(n = 8\).

The second subtask will have two files with \(n = 6\) and \(n = 12\).

\section*{Gibberish}

You should basically just ignore the rest of this. These are just demonstrations for different latex commands available.

If is \(n\) an integer and is a \(p\) prime factor of \(n\), the multiplicity of \(p\) in \(n\) is defined as the number of times \(p\) can divide \(n\). For example, the multiplicity of $2$ in $24$ is $3$ since you can divide $24$ up to three times, and no more. Let's denote the multiplicity of $p$ in $n$ by $o_p(n)$. For example \(o_2(24) = 3\), 

You compute the function $f(n)$ defined as:

$$f(n) = \prod_{\substack{p \\ p \geq 2 \text{ is a prime dividing } n}} p^{o_p(n)}$$

Then you compute the number $k$ defined as:

\[\int_{0}^{2\sqrt{5}} x \, dx\]

\large{\bf{Bounds:}}
$$ 20 \le n \le 20 $$

This is how you make LaTeX stuff:

This is 100\% worth \$\$\$ bills.

This is \textbf{bold} text. So \bf{brave}.

This is \textit{italic} text. So \it{italian}.

This is \texttt{monospace} text. So \tt{teletype}.

This is \underline{underlined} text. So \emph{emphasized}.

This is \sout{strikethrough'ed} text.

This is \textsc{uppercase} text.

Here are all the font sizes.

\tiny{tiny} or \scriptsize{scriptsize}

\small{small}

\normalsize{normalsize}

\large{large}

\Large{Large}

\LARGE{LARGE}

\huge{huge}

\Huge{Huge}

\HUGE{HUGE}

Here is a link \url{http://example.com}

Here is a nicer \href{http://example.com}{link}

You can also nest all of these together. \href{http://example.com}{\HUGE{\sout{\tt{\bf\it{{link}}}}}}

Here is a picture for your troubles \includegraphics{/tasks/batch-demo/attachments/path/to/chocolate-hills.jpg}

Now let me test some stuff with center!

\begin{center}
    \includegraphics[scale=0.5]{/tasks/batch-demo/attachments/path/to/chocolate-hills.jpg}
\end{center}


\begin{center}
    Some veeeeeeeeeeeeeeeeeeeeeeeeeeeeeeeeeeeeeeeeeeeeeeeeeeeery looooooooooooooooooooooooooooooooooooooooong \textbf{text} here

    \includegraphics[scale=0.5]{/tasks/batch-demo/attachments/path/to/chocolate-hills.jpg}


    Some veeeeeeeeeeeeeeeeeeeeeeeeeeeeeeeeeeeeeeeeeeeeeeeeeeeery looooooooooooooooooooooooooooooooooooooooong \textbf{text} here
\end{center}

\begin{center}
    short \textbf{text} here

    \includegraphics[scale=0.5]{/tasks/batch-demo/attachments/path/to/chocolate-hills.jpg}


    short \textbf{text} here
\end{center}


\begin{center}
    short \textbf{text} here
\end{center}